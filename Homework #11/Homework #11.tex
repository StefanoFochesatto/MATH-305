% !TEX TS-program = pdflatexmk
\documentclass[12pt]{amsart}

%\usepackage[parfill]{parskip}    % Activate to begin paragraphs with an empty line rather than an indent
\usepackage{gensymb}
\usepackage{float}
\usepackage[margin=1in]{geometry}

\usepackage{amsmath,amssymb,amsthm,latexsym,graphicx}
\usepackage[normalem]{ulem}
\usepackage{setspace} %used for doublespacing, etc.
\usepackage{hyperref}
\usepackage{cancel}
\usepackage[dvipsnames,usenames]{color}
\usepackage[all]{xy}
\usepackage{fancyhdr}
\pagestyle{fancy}
	\renewcommand{\headrulewidth}{0.5pt} % and the line
	\headsep=1cm
	
\DeclareGraphicsRule{.tif}{png}{.png}{`convert #1 `dirname #1`/`basename #1 .tif`.png}

%Some useful environments.
\newtheorem{theorem}{Theorem}
\newtheorem{corollary}[theorem]{Corollary}
\newtheorem{conjecture}[theorem]{Conjecture}
\newtheorem{lemma}[theorem]{Lemma}
\newtheorem{proposition}[theorem]{Proposition}
\newtheorem{definition}[theorem]{Definition}
\newtheorem{example}[theorem]{Example}
\newtheorem{axiom}{Axiom}
\theoremstyle{remark}
\newtheorem{remark}{Remark}
\newtheorem*{exercise}{Exercise}%[section]

%For GeoGebra
\usepackage{pgf,tikz,pgfplots}
\pgfplotsset{compat=1.15}
\usepackage{mathrsfs}
\usetikzlibrary{arrows}
\newcommand{\degre}{\ensuremath{^\circ}}


%Some useful shortcuts for our favorite sets of numbers
%Note, you can use these WITHOUT entering math mode

\newcommand{\RR}{\ensuremath{\mathbb R}} 
\newcommand{\NN}{\ensuremath{\mathbb N}}
\newcommand{\ZZ}{\ensuremath{\mathbb Z}}
\newcommand{\QQ}{{\ensuremath\mathbb Q}}
\newcommand{\CC}{\ensuremath{\mathbb C}}
\newcommand{\EE}{{\ensuremath\mathbb E}}

%Some useful shortcuts for formatting lists
\newcommand{\bc}{\begin{center}}
\newcommand{\ec}{\end{center}}
\newcommand{\be}{\begin{enumerate}}
\newcommand{\ee}{\end{enumerate}}
\newcommand{\bi}{\begin{itemize}}
\newcommand{\ei}{\end{itemize}}

%Some useful shortcuts for formatting mathematical symbols
\newcommand{\ol}[1]{\overline{#1}}
\newcommand{\oimp}[1]{\overset{#1}{\iff}} %labeled iff symbol
\newcommand{\bv}[1]{\ensuremath{ \vec{\mathbf{#1}}} } %makes a vector.
\newcommand{\mc}[1]{\ensuremath{\mathcal{#1}}} %put something in caligraphic font
\newcommand{\bsl}[1]{\texttt{\symbol{92}{\em #1}}} %for backslashes.
\newcommand{\normale}{\trianglelefteq}
\newcommand{\normal}{\triangleleft}

%Commenting tools --- You can ignore these, but if you have a question about latex and send me your source file, I'll use them to explain stuff to you.
\newcommand{\mpg}[1]{\marginpar{ #1}} %to put comments in margins
\usepackage{soul}
\definecolor{highlight}{rgb}{1,0.6,0.6}
\sethlcolor{highlight}
\newcommand{\hlm}[1]{\colorbox{highlight}{$\displaystyle #1$}}
\newtheoremstyle{mycomment}{\topsep}{-0in}{\small \itshape \sffamily}{}{\small \itshape\sffamily}{:}{.5em}{}
\theoremstyle{mycomment}
\newtheorem*{acomment}{\color{BrickRed}{Comment}}
\newcommand{\com}[1]{{\color{OliveGreen}\begin{acomment}{#1} 
\end{acomment}\noindent}}
\newcommand{\red}[1]{{\color{BrickRed} #1}}
\newcommand{\blue}[1]{{\color{MidnightBlue}#1}}
\newcommand{\green}[1]{{\color{OliveGreen}#1}}
\newcommand{\mwrong}[2]{\red{\cancel{#1}}\green{#2}}
\newcommand{\wrong}[2]{\red{\sout{#1}}\green{#2}}
\definecolor{OliveGreen}{rgb}{.3,.5,.2}
\definecolor{MidnightBlue}{rgb}{.3,.4,.6}
\newcommand{\pts}[1]{\hfill\blue{{#1}/5}}

%%%%%%%%%%%%%%%%%%%%%%%%%%%%%%%%%%%%%%%%%%%%%%%%%%%%%%%%%%%%%%%%%%%%%%%%%%%%%%%%%%%%%%%%%%%%%%%%%%%%%%%%%%%%%%

\chead{MATH F305}
\pagestyle{fancy}
%Modify these items:
\rhead{\emph{Stefano Fochesatto}}
\lhead{\emph{HW \#11 --- 3/25/20}}


\begin{document}
\thispagestyle{fancy}

\begin{enumerate}

\item(Problem $\Box 6.1$ )\\
\\
\setstretch{1.3} If a full mesh $\mathcal{M}$ is not empty, then $\mathcal{M}$ contains at least six elements.

\textbf{Answer:} Suppose $M$ is a full mesh. Note that $M$ must have at least 3 points because it is a mesh, this was proven in page 78 problems 3 and 4. Since $M$ is defined as a full mesh, for every pair of distinct points $P$,$Q$ the line $\overline{PQ}$ must also be in the mesh. It then follows that for each distinct pair of points there must also exist a distinct line. Counting the pairs for the smallest case full mesh, 
\setstretch{2.5}
\begin{equation}
{3 \choose 2} = 3.
\end{equation}
Therefore $M$ must also have at least 3 lines. Thus $M$ must have at least six elements. 
\setstretch{2.5}
\vspace{1in}

\item(Problem $\Box 6.2$ )\\
\\
\setstretch{1.3} If a full mesh $\mathcal{M}$ contains at least four points, then $\mathcal{M}$ contains infinitely many elements.

\textbf{Answer:} Consider the following full mesh with 4 points, and 4 lines.
\setstretch{2.5}

\begin{figure}[H]
\caption{}
\centering
\includegraphics[width=.5\textwidth]{"trimesh".png}
\end{figure}


\vspace{1in}

\item(Page $23$ $\#8$ )\\
\\
\setstretch{1.3} If two circles intersect, the straight line which joins their centers is the right bisector of their common chord. (Note, a circle is the set of points equidistant from a specified point called the centre.)

\textbf{Answer:} Let 2 circles with center points $A$ and $C$ intersect with common chord $\overline{EF}$. Let point $G$ be the intersection of $\overline{AC}$  and $\overline{EF}$. Consider the following figure.
\setstretch{2.5}

\begin{figure}[H]
\caption{}
\centering
\includegraphics[width=\textwidth]{"circle".png}
\end{figure}

Since lines $\overline{AE}$ and $\overline{AF}$ are radii of circle we know that $\overline{AE} \cong \overline{AF}$ and similarly $\overline{CE} \cong \overline{CF}$. Thus it follows that  $\triangle AEC \cong \triangle AFC$ by SSS which gives us that $\angle GAF \cong \angle GAE$. It then follows that $\triangle GAE \cong \triangle GAF$ by SAS. Therefore $\angle EGA = \angle FGA$. We know that $\angle EGA = \angle FGA = \angle EGC = \angle FGC$ by Vertical Angle Theorem. The only way to have 4 equivalent angles divide $360^ {\degree}$ is if they all equal $90^ {\degree}$.
\setstretch{2.5}
\vspace{1in}

\item(Page $38$  $\#1$ )\\
\\
\setstretch{1.3} If the bisector of an exterior angle of a triangle is parallel to the opposite
side, the triangle must have two of its angles equal.

\textbf{Answer:} Consider triangle $\triangle ABC$ with exterior angle $\angle ACE$ such that $\overline{CD}$ bisects $\angle ACE$. Also suppose that $\overline{CD}$ is parallel to $\overline{BA}$.
\setstretch{2.5} 

\begin{figure}[H]
\caption{}
\centering
\includegraphics[width=\textwidth]{"tri".png}
\end{figure}

Since $\overline{CD}$ is a bisector of angle $\angle ACE$ we know that angles $\angle ACD = \angle DCE$. Since $\overline{CD}$ is parallel to $\overline{BA}$ with transversal $\overline{AC}$ we know by alternate interior angles, that $\angle BAC = \angle ACD$. Considering the same set of parallel lines but with transversal $\overline{BE}$ we know by corresponding angles that $\angle ABC = \angle DCE$. Through substitution, 
\begin{align}
\angle ACD &= \angle DCE,\\
\angle BAC &=\angle ABC.
\end{align}
Thus $\triangle ABC$ has two equal angles. 

\setstretch{2.5}
\vspace{1in}

\end{enumerate}
 \end{document}
 \end
