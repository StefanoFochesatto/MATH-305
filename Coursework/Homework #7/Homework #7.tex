% !TEX TS-program = pdflatexmk
\documentclass[12pt]{amsart}

%\usepackage[parfill]{parskip}    % Activate to begin paragraphs with an empty line rather than an indent

\usepackage[margin=1in]{geometry}

\usepackage{amsmath,amssymb,amsthm,latexsym,graphicx}
\usepackage[normalem]{ulem}
\usepackage{setspace} %used for doublespacing, etc.
\usepackage{hyperref}
\usepackage{cancel}
\usepackage[dvipsnames,usenames]{color}
\usepackage[all]{xy}
\usepackage{fancyhdr}
\pagestyle{fancy}
	\renewcommand{\headrulewidth}{0.5pt} % and the line
	\headsep=1cm
	
\DeclareGraphicsRule{.tif}{png}{.png}{`convert #1 `dirname #1`/`basename #1 .tif`.png}

%Some useful environments.
\newtheorem{theorem}{Theorem}
\newtheorem{corollary}[theorem]{Corollary}
\newtheorem{conjecture}[theorem]{Conjecture}
\newtheorem{lemma}[theorem]{Lemma}
\newtheorem{proposition}[theorem]{Proposition}
\newtheorem{definition}[theorem]{Definition}
\newtheorem{example}[theorem]{Example}
\newtheorem{axiom}{Axiom}
\theoremstyle{remark}
\newtheorem{remark}{Remark}
\newtheorem*{exercise}{Exercise}%[section]

%For GeoGebra
\usepackage{pgf,tikz,pgfplots}
\pgfplotsset{compat=1.15}
\usepackage{mathrsfs}
\usetikzlibrary{arrows}
\newcommand{\degre}{\ensuremath{^\circ}}


%Some useful shortcuts for our favorite sets of numbers
%Note, you can use these WITHOUT entering math mode

\newcommand{\RR}{\ensuremath{\mathbb R}} 
\newcommand{\NN}{\ensuremath{\mathbb N}}
\newcommand{\ZZ}{\ensuremath{\mathbb Z}}
\newcommand{\QQ}{{\ensuremath\mathbb Q}}
\newcommand{\CC}{\ensuremath{\mathbb C}}
\newcommand{\EE}{{\ensuremath\mathbb E}}

%Some useful shortcuts for formatting lists
\newcommand{\bc}{\begin{center}}
\newcommand{\ec}{\end{center}}
\newcommand{\be}{\begin{enumerate}}
\newcommand{\ee}{\end{enumerate}}
\newcommand{\bi}{\begin{itemize}}
\newcommand{\ei}{\end{itemize}}
\usepackage{float}


%Some useful shortcuts for formatting mathematical symbols
\newcommand{\ol}[1]{\overline{#1}}
\newcommand{\oimp}[1]{\overset{#1}{\iff}} %labeled iff symbol
\newcommand{\bv}[1]{\ensuremath{ \vec{\mathbf{#1}}} } %makes a vector.
\newcommand{\mc}[1]{\ensuremath{\mathcal{#1}}} %put something in caligraphic font
\newcommand{\bsl}[1]{\texttt{\symbol{92}{\em #1}}} %for backslashes.
\newcommand{\normale}{\trianglelefteq}
\newcommand{\normal}{\triangleleft}

%Commenting tools --- You can ignore these, but if you have a question about latex and send me your source file, I'll use them to explain stuff to you.
\newcommand{\mpg}[1]{\marginpar{ #1}} %to put comments in margins
\usepackage{soul}
\definecolor{highlight}{rgb}{1,0.6,0.6}
\sethlcolor{highlight}
\newcommand{\hlm}[1]{\colorbox{highlight}{$\displaystyle #1$}}
\newtheoremstyle{mycomment}{\topsep}{-0in}{\small \itshape \sffamily}{}{\small \itshape\sffamily}{:}{.5em}{}
\theoremstyle{mycomment}
\newtheorem*{acomment}{\color{BrickRed}{Comment}}
\newcommand{\com}[1]{{\color{OliveGreen}\begin{acomment}{#1} 
\end{acomment}\noindent}}
\newcommand{\red}[1]{{\color{BrickRed} #1}}
\newcommand{\blue}[1]{{\color{MidnightBlue}#1}}
\newcommand{\green}[1]{{\color{OliveGreen}#1}}
\newcommand{\mwrong}[2]{\red{\cancel{#1}}\green{#2}}
\newcommand{\wrong}[2]{\red{\sout{#1}}\green{#2}}
\definecolor{OliveGreen}{rgb}{.3,.5,.2}
\definecolor{MidnightBlue}{rgb}{.3,.4,.6}
\newcommand{\pts}[1]{\hfill\blue{{#1}/5}}
\usepackage{scalerel,amssymb}

\def\mcirc{\mathbin{\scalerel*{\bigcirc}{t}}}
\def\msquare{\mathord{\scalerel*{\Box}{gX}}}

\chead{MATH F305}
\pagestyle{fancy}
%Modify these items:
\rhead{\emph{Stefano Fochesatto}}
\lhead{\emph{HW \#7 --- 2/17/20}}


\begin{document}

\thispagestyle{fancy}
\setstretch{2.5} %Use for 2.5 spacing. This is the default setting for homework in this class... so I have space to write comments.
%\doublespacing %Use for double spacing
%\singlespacing %Use for single spacing

\begin{exercise}[\msquare 4.2.1] Prove the converse of Ceva's Theorem. (HINT: Let $X=AD*BE$ and $F'= CX*AB$ Use Ceva's Theorem to say something about the situation involving $F'$, then use that to sho $F = F'$
\begin{proof} Let $\triangle ABC$ be a triangle, and let $D$, $E$, and $F$ be on the lines $BC$, $CA$, and $AB$ respectively. 
\begin{figure}[H]
\caption{Ceva's Converse}
\centering
\includegraphics[width=\textwidth]{"CevaConverse".png}
\end{figure}
Suppose the following,
\begin{equation*}
\frac{|AF'|}{|F'B|} \frac{|BD|}{|DC|} \frac{|CE|}{|EA|} = 1
\end{equation*}
We know that two lines that are not parallel must intersect at a point. Let $AD$ and $BE$ intersect at point $X$. Now Consider a third line that intersects point $X$, $CF$. By Ceva's Theorem we know that 
\begin{equation*}
\frac{|AF|}{|FB|} \frac{|BD|}{|DC|} \frac{|CE|}{|EA|} = 1
\end{equation*}
Yet we supposed that,
\begin{equation*}
\frac{|AF'|}{|F'B|} \frac{|BD|}{|DC|} \frac{|CE|}{|EA|} = 1
\end{equation*}
Therefore by cancellation, 
\begin{equation*}
\frac{|AF|}{|FB|} = \frac{|AF'|}{|F'B|} .
\end{equation*}
and thus $F' = F$ therefore lines $AD$, $BE$, and $FC$ are concurrent.


\end{proof}
\end{exercise}

\begin{exercise}[\msquare 4.2.2] Prove Menelaus’s theorem, and while doing so, determine the constant that goes in the blank. (Hint: As in the proof of Ceva’s theorem, draw a line through $C$ parallel to $AB$ and find similar triangles again. See Figure 4.6.)\\


\begin{proof}[Answer:] 
Let $\triangle ABC$ be a triangle, and let a transversal line intersect sides $BC$, $CA$, and $AB$ at $D$, $E$, and $F$ respectively such that $D$, $E$, and $F$ are distinct from $A$, $B$, and $C$. 

\begin{figure}[H]
\caption{Setup For Menelaus’s Theorem}
\centering
\includegraphics[width=\textwidth]{"Mene2".png}
\end{figure}

Consider the triangles formed from perpendicular lines that go from the vertices of the triangle to the traversal.

\begin{figure}[H]
\caption{Menelaus’s Theorem with Right Triangles}
\centering
\includegraphics[width=\textwidth]{"Mene1".png}
\end{figure}


 So we can make a few observations about the triangles formed in the scene. Since they are all right triangles we know that $\triangle IBD \sim \triangle CGD$, $\triangle CGE \sim \triangle AHE$, and $\triangle IFB \sim \triangle HFA$, all by AA similarity (congruent angles are denoted by color, $\triangle IFB$ and $\triangle HFA$ share an angle). Then we can make the following statements about the ratio of edges,
\begin{equation*}
\frac{BD}{CD} = \frac{BI}{CG}
\end{equation*}
\begin{equation*}
\frac{CE}{AE} = \frac{CG}{AH}
\end{equation*}
\begin{equation*}
\frac{FA}{FG} = \frac{AH}{BI}
\end{equation*}
Note that,
\begin{equation*}
\frac{BI}{CG}\frac{CG}{AH}\frac{AH}{BI} = 1
\end{equation*}
Therefore by substitution,
\begin{equation*}
\frac{BD}{CD}\frac{CE}{AE} \frac{FA}{FG}  = 1
\end{equation*}




















\end{proof}
\end{exercise}


 \end{document}
 \end

  