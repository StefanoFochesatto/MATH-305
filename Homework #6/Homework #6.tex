% !TEX TS-program = pdflatexmk
\documentclass[12pt]{amsart}

%\usepackage[parfill]{parskip}    % Activate to begin paragraphs with an empty line rather than an indent

\usepackage[margin=1in]{geometry}
\usepackage{float}
\usepackage{scalerel,amssymb}
\def\mcirc{\mathbin{\scalerel*{\circ}{j}}}
\def\msquare{\mathord{\scalerel*{\Box}{gX}}}

\usepackage{amsmath,amssymb,amsthm,latexsym,graphicx}
\usepackage[normalem]{ulem}
\usepackage{setspace} %used for doublespacing, etc.
\usepackage{hyperref}
\usepackage{cancel}
\usepackage[dvipsnames,usenames]{color}
\usepackage[all]{xy}
\usepackage{fancyhdr}
\pagestyle{fancy}
	\renewcommand{\headrulewidth}{0.5pt} % and the line
	\headsep=1cm
	
\DeclareGraphicsRule{.tif}{png}{.png}{`convert #1 `dirname #1`/`basename #1 .tif`.png}

%Some useful environments.
\newtheorem{theorem}{Theorem}
\newtheorem{corollary}[theorem]{Corollary}
\newtheorem{conjecture}[theorem]{Conjecture}
\newtheorem{lemma}[theorem]{Lemma}
\newtheorem{proposition}[theorem]{Proposition}
\newtheorem{definition}[theorem]{Definition}
\newtheorem{example}[theorem]{Example}
\newtheorem{axiom}{Axiom}
\theoremstyle{remark}
\newtheorem{remark}{Remark}
\newtheorem*{exercise}{Exercise}%[section]

%For GeoGebra
\usepackage{pgf,tikz,pgfplots}
\pgfplotsset{compat=1.15}
\usepackage{mathrsfs}
\usetikzlibrary{arrows}
\newcommand{\degre}{\ensuremath{^\circ}}


%Some useful shortcuts for our favorite sets of numbers
%Note, you can use these WITHOUT entering math mode

\newcommand{\RR}{\ensuremath{\mathbb R}} 
\newcommand{\NN}{\ensuremath{\mathbb N}}
\newcommand{\ZZ}{\ensuremath{\mathbb Z}}
\newcommand{\QQ}{{\ensuremath\mathbb Q}}
\newcommand{\CC}{\ensuremath{\mathbb C}}
\newcommand{\EE}{{\ensuremath\mathbb E}}

%Some useful shortcuts for formatting lists
\newcommand{\bc}{\begin{center}}
\newcommand{\ec}{\end{center}}
\newcommand{\be}{\begin{enumerate}}
\newcommand{\ee}{\end{enumerate}}
\newcommand{\bi}{\begin{itemize}}
\newcommand{\ei}{\end{itemize}}

%Some useful shortcuts for formatting mathematical symbols
\newcommand{\ol}[1]{\overline{#1}}
\newcommand{\oimp}[1]{\overset{#1}{\iff}} %labeled iff symbol
\newcommand{\bv}[1]{\ensuremath{ \vec{\mathbf{#1}}} } %makes a vector.
\newcommand{\mc}[1]{\ensuremath{\mathcal{#1}}} %put something in caligraphic font
\newcommand{\bsl}[1]{\texttt{\symbol{92}{\em #1}}} %for backslashes.
\newcommand{\normale}{\trianglelefteq}
\newcommand{\normal}{\triangleleft}

%Commenting tools --- You can ignore these, but if you have a question about latex and send me your source file, I'll use them to explain stuff to you.
\newcommand{\mpg}[1]{\marginpar{ #1}} %to put comments in margins
\usepackage{soul}
\definecolor{highlight}{rgb}{1,0.6,0.6}
\sethlcolor{highlight}
\newcommand{\hlm}[1]{\colorbox{highlight}{$\displaystyle #1$}}
\newtheoremstyle{mycomment}{\topsep}{-0in}{\small \itshape \sffamily}{}{\small \itshape\sffamily}{:}{.5em}{}
\theoremstyle{mycomment}
\newtheorem*{acomment}{\color{BrickRed}{Comment}}
\newcommand{\com}[1]{{\color{OliveGreen}\begin{acomment}{#1} 
\end{acomment}\noindent}}
\newcommand{\red}[1]{{\color{BrickRed} #1}}
\newcommand{\blue}[1]{{\color{MidnightBlue}#1}}
\newcommand{\green}[1]{{\color{OliveGreen}#1}}
\newcommand{\mwrong}[2]{\red{\cancel{#1}}\green{#2}}
\newcommand{\wrong}[2]{\red{\sout{#1}}\green{#2}}
\definecolor{OliveGreen}{rgb}{.3,.5,.2}
\definecolor{MidnightBlue}{rgb}{.3,.4,.6}
\newcommand{\pts}[1]{\hfill\blue{{#1}/5}}
\newcommand{\bproof}{\begin{proof}\setstretch{1.5}}
\newcommand{\eproof}{\end{proof}\singlespacing}
\chead{MATH F305}
\pagestyle{fancy}
%Modify these items:
\rhead{\emph{Stefano Fochesatto}}
\lhead{\emph{HW \#6 --- 2/10/20}}


\begin{document}

\thispagestyle{fancy}
%\setstretch{2.5} %Use for 2.5 spacing. This is the default setting for homework in this class... so I have space to write comments.
%\doublespacing %Use for double spacing
%\singlespacing %Use for single spacing

\begin{exercise}[$\msquare$ 4.1.1] Prove that if $A, B, C$ and $D$ are collinear points in that order and $\overline{AB} \cong \overline{CD}$ then $\overline{AC} \cong \overline{BD}$
\bproof Suppose  $A, B, C$ and $D$ are collinear points in that order and $\overline{AB} \cong \overline{CD}$, 

\begin{figure}[H]
\caption{Scene described in 4.1.1}
\centering
\includegraphics[width=.5\textwidth]{"line".png}
\end{figure}
	





Consider the following, by the Sum of Lengths axiom,
\begin{equation*}
||\overline{AB}||+||\overline{BC}|| = ||\overline{AC}||
\end{equation*}
\begin{equation*}
 ||\overline{BC}||+||\overline{CD}|| = ||\overline{BD}||
\end{equation*}
Solving for $||\overline{AB}||$ and $||\overline{CD}||$
\begin{equation*}
||\overline{AC}||-||\overline{BC}|| = ||\overline{AB}||
\end{equation*}
\begin{equation*}
||\overline{BD}||-||\overline{BC}|| = ||\overline{CD}||
\end{equation*}
By congruence we know, 
\begin{equation*}
||\overline{AB}|| = ||\overline{CD}||
\end{equation*}
By Substitution,
\begin{align*}
||\overline{AC}||-||\overline{BC}| &= ||\overline{BD}||-||\overline{BC}||\\
||\overline{AC}|| &= ||\overline{BD}||
\end{align*}
Thus by measurement $\overline{AC} \cong \overline{BD}$.
\eproof 
\end{exercise}


\begin{exercise}[$ \msquare$ 4.1.2] Prove that if a pair of parallel lines is cut by a transversal, the corresponding angles are congruent\\
\begin{figure}[H]
\caption{Scene described in 4.1.2}
\centering
\includegraphics[width=.5\textwidth]{"Parallel".png}
\end{figure}
\bproof[Answer:] (Contradiction)
Suppose that a pair of parallel lines is cut by a transversal, and the corresponding angles are not congruent. Since the corresponding angles are not congruent we know that $\angle EID \neq \angle EJF$.\\

Case 1: Suppose $||\angle EID|| > ||\angle EJF||$. We know that,
\begin{equation*}
||\angle EID || + || \angle HID ||  = 180^o.
\end{equation*}
By substitution we know that,
\begin{equation*}
||\angle EJF || + || \angle HID ||  < 180^o.
\end{equation*}
By Euclid's parallel postulate\\

If a line segment intersects two straight lines forming two interior angles on the same side that sum to less than two right angles, then the two lines, if extended indefinitely, meet on the side on which the angles sum to less than two right angles. (p.48)\\

Since $\angle EJF$ and $\angle HID$ are two interior angles on the same side of the transversal that sum to less than $180^o$ we know that the two lines $\overline{GF}$ and $\overline{CD}$ must intersect on the same side of the transversal. Thus $\overline{GF}$ and $\overline{CD}$ are both parallel and not parallel.

\eproof
\end{exercise}

\begin{exercise}[ $\msquare$ 4.1.3]  Prove that the measurements of three interior angles of a triangle sum to $180^o$

\begin{figure}[H]
\caption{Scene described in 4.1.3}
\centering
\includegraphics[width=.66\textwidth]{"Triangle".png}
\end{figure}
	

\bproof Consider $\triangle ABC$. Now construct a line that is parallel to one side of the triangle and also incident to one point of the triangle. Using our diagram we can say that $\overline{DE} || \overline{AC}$. Note that the lines $\overline{AB}$ and $\overline{CB}$ are by definition, transversal lines. We have just proved that the corresponding angles of a pair of lines cut by a transversal are congruent, therefore we can make the following claims,
\begin{equation*}
||\angle ABE || = ||\angle HAC ||
\end{equation*}
\begin{equation*}
||\angle CBD || = ||\angle ICA ||
\end{equation*}
We can also make the following claims because we know that the sum of all the angles on a single side of a line must add up to $180^o$,
\begin{equation*}
||\angle ABD || + ||\angle ABC || + ||\angle CBE || = 180^o  
\end{equation*}
\begin{equation*}
||\angle HAC || + ||\angle CAB || = 180^o  
\end{equation*}
\begin{equation*}
||\angle ICA || + ||\angle BCA || = 180^o  
\end{equation*}
With all these claims, 
WTS that $||\angle CAB || = ||\angle ABD || $.\\
Note,
\begin{equation*}
180^o - ||\angle ABD || =  ||\angle ABE ||
\end{equation*}
Also note, 
\begin{equation*}
180^o - ||\angle CAB || =  ||\angle HAC ||
\end{equation*}
Since $||\angle ABE || = ||\angle HAC ||$ we know that by cancellation, 
\begin{equation*}
||\angle CAB || = ||\angle ABD ||
\end{equation*}\\

By a similar argument
WTS that $||\angle BCA || = ||\angle CBE || $.\\
Note,
\begin{equation*}
180^o - ||\angle CBE || =  ||\angle CBD ||
\end{equation*}
Also note, 
\begin{equation*}
180^o - ||\angle BCA || =  ||\angle ICA ||
\end{equation*}
Since $ ||\angle CBD || = ||\angle ICA ||$  we know that by cancellation, 
\begin{equation*}
||\angle BCA || = ||\angle CBE ||
\end{equation*}

Thus by substitution into  $||\angle ABD || + ||\angle ABC || + ||\angle CBE || = 180^o$ it must be true that,
\begin{equation*}
||\angle CAB || + ||\angle ABC || + ||\angle BCA || = 180^o  
\end{equation*}


 

\eproof 
\end{exercise}


 \end{document}
 \end

  